
%\documentclass[a4paper,10pt]{paper}
\documentclass{llncs}
\usepackage[utf8]{inputenc}
\usepackage{hyperref}
\usepackage{graphicx}
\usepackage{wrapfig}
\usepackage{listings}
\usepackage{float}
\usepackage{bytefield}
%\usepackage{fullpage}
\usepackage{algorithm}
\usepackage{algpseudocode}
\usepackage{amsmath}
%\usepackage{todonotes}

%opening
\title{LickNet: Software Tools Collection for Guitar Licks Networks}
\author{Matteo Martelli}
\institute{
	University of Bologna\\ 
	\email{matteo.martelli9@studio.unibo.it}
}
\begin{document}

\maketitle

\begin{abstract}
The purpose of the project described in this document is to provide a
method to analyse the relations between musical note patterns 
in guitar solos.
Given the large amount of guitar solo data available in the Web, it has been possible to
 develop an algorithm able to generate a network of guitar solos.
An empirical study has then been conducted in
order to identify some well known characteristics of the generated
networks, which showed that it seems to satisfy the small world
property\cite{sw-watts}. Moreover, with the intent to provide an application
for the generated networks, two interesting tools have been developed:
a Licks Classifier, useful to identify to which artist a set of
guitar licks belongs to, and a Licks Generator, useful to generate new
licks which are intended to reflect a particular guitarist's style. 
Thanks to the intuitiveness and surprising effectiveness of implementing such
relatively simple algorithms, the initial expectations about the project
described in this document,
have been satisfied.
\end{abstract}
\section{Introduction}
Musicians are often fascinated by how other musicians compose or
improvise their music. A musician may also often feel some kind of relation between his
music and some other musician's music.
Even non expert music enthusiasts often wonder what makes a 
musician so unique and easily recognisable among many, or what makes a
musician so similar to another. This frequently happens with guitarists for example, when people recognize 
a \emph{Santana} guitar solo or a \emph{Hendrix} guitar solo just by listening 
 to a few notes, or when people
say that \emph{Steve Vai} sounds similar to \emph{Joe
Satriani}.\footnote{\emph{Last.fm}, a
collaborative website, assigns a "super" level of similarity between Steve Vai
and Joe Satriani. This may very well be due to the fact that Steve Vai was his student.}.
One may also think that this is due to the fact that there are repetitions among some note patterns
in a musician's repertoire or that similar note patterns emerge from their common musical influence or musical tastes.

Thus the idea for \emph{LickNet} emerged from my effort to analyse the relations (modeled as networks) between note patterns in guitar
solos (licks) as described in this document. The choice to focus on guitar solos has been made 
because of the large amount of data available online but my work may easily
be extended to other similar kinds of data. For example drum patterns, whole compositions, improvisations or even jam sessions may be considered. 

It has been chosen to model
the data elements and their relations within a network as the
analytical methods in graph theory are very useful. The area of scientific research
that is usually interested in the empirical examination of real-world networks
 is the study of \emph{Complex Networks}. Different approaches
and mathematical methods pertaining to the Complex Networks
field of study have been considered and will be discussed in the next sections; 
including graph theory and small-world network
comparison\cite{complex-networks}.

As said before, in addition to the network analysis tools, LickNet provides other
features like
the \emph{Lick Classifier} and the \emph{Lick Generator}. These
functionalities are still in the early stages of development but they can be a starting point
for some practical applications; e.g. to simply understand and replicate some guitarist's playing style or to
create a full-fledged artificial agent able to ``play'' music and ``interact'' with human players.

\section{Related Works}
Several empirical studies about music have already been made using network analysis.
In \cite{music-genres}, networks have been
used to model the correlation among musical tastes of
different listener. In addition, the same paper shows a technique to
create a network of musical groups and artists which are connected by
the similarity of their audience. A similar approach has been used in
other remarkable works \cite{artists-network}\cite{metal-network}.
The authors of \cite{sw-words} introduced a method to create a network of human language words, linking
them through a correlation parameter such as their relative distance in
a sentence, choosing a value of 2 as the maximum distance for forming links. This
means that two words can be linked together if in a sentence one word is the
successor or the successor of the successor of the other word. They
made this choice for two reasons:
\begin{enumerate}
\item Many co-occurrences of
words take place at a distance of one, e.g. `red flowers' (adjective-noun), 
`stay here' (verb-adverb), `getting dark' (verb-adjective), etc. 
\item Many co-occurrences of words take place at a distance of two, e.g.
`hit the ball' (verb-object),
`Mary usually cries' (subject-verb), `Live in Boston' (verb-noun), etc..
\end{enumerate}
In the  section \ref{sec:networks} it will be pointed out  
that a similar graph creation procedure has been used in this work, but
with the maximum distance for forming links of 1. This because it is not
trivial
to determine a multi-step correlation value between musical notes in a
notes lick.
Use of a maximum distance greater than 1 may be tested in future versions.

\section{LickNet}
As stated before, LickNet offers the possibility to study the
relations among note patterns in guitar solos through the creation
and analysis of guitar licks networks. It comprises software tools with which
it is possible to operate on these networks.
These operations are currently two, a classifier and a generator of
guitar licks.\\
Before getting into the software components, the next section is going
to focus on how data is collected, structured and processed. 

\section{Data Representation}
As many guitarists may already know, there are plenty of websites that
store guitar sheets and tablatures. One of them is \url{www.ultimate-guitar.com}
that counts more than 800000 sheets, which are mostly written
in the guitar tablature notation. Guitar players usually choose this
notation for its ease of use\footnote{Guitar tablature removes the
requirement for the player to remember the association between each
note and the corresponding finger position on the fretboard since they are directly
represented in numbers (which fret) for each string.}
but all this is instrument-specific. Also, guitar tablature is not standardized and
different publishers adopt different conventions. 
This means that a
guitar tablature can be understood only by guitar players and its
conversion to the standard notation or formal interpretation may result
wearing. 
For example the semantic of a guitar tablature changes even if
a guitar is not standard tuned.\\
Fortunately there are various computer programs available for writing
tablature. One of the most frequently used is \emph{Guitar Pro} and the
\emph{Ultimate
Guitar} servers store a large amount of tablature files encoded in this particular format.
Another interesting software is \emph{TuxGuitar}, a free and open source
tablature editor that also supports the ability to import and export
Guitar Pro files\cite{tuxguitar}. Moreover its source code can be freely
re-used and adapted to any other application. LickNet, in fact, includes
the TuxGuitar library with the purpose of importing tablature files retrieved
online.\\
This way tablature can be interpreted by software and a guitar solo
can be represented as a sequence of notes, each having the following properties:
\begin{itemize}
	\item \textbf{String}: at which string the note is played
	\item \textbf{Value}: at which fret the note is played
	\item \textbf{Effects}: a collection of effects such as bending, vibrato,
		harmonic, etc.
	\item \textbf{Duration}: the duration of the note, represented
as a fraction.
\end{itemize}
This is how the TuxGuitar library represents the notes, but some
adaptation and additional piece of 
information is needed for my purpose. It has been decided
to represent a single note in LickNet as the following:
\begin{itemize}
	\item \textbf{Base Note}: the value of the musical note. It is
a value in the range of [0,12] irrespective of the octave. The 0th
note is assigned to C, the 11th to B, the last one, and the 12th is the rest.\footnote{A
rhythmic period of silence between tones.}
	\item \textbf{Octave}: the octave of the note. The note pitch $p$ is given
by the equation $p = o \times 12 + b $, where $o$ is the octave and $b$
is the base note.
	\item \textbf{Time}: it's a floating point value in the range [0,1] indicating the duration of
the note. The dotted and double dotted notes are also considered, as are
the triplets. For example a quaver dotted note would have a
time value of $\frac{1}{8} + \frac{1}{16} = 0.1875$ and a semi-quaver triplet note
would have the time value of $\frac{1}{16} \times \frac{2}{3} \simeq 0.042$.
	\item \textbf{Bend Distance}: if a note has a bending effect, the bending
distance is the offset from the base note to the bended note. For
example if a F note (base note = 5) is bended to G than the distance is
2, but if it is bended to F\# (half step bending) then the distance is 1. It
is possible that the bending effect is defined with more complex
properties: it may be defined with the \emph{bending and release}
property or with the \emph{bending,
release and bending again} property. Thus it has been decided to calculate
the bending distance has the average bending distance.
	\item \textbf{Node Key}: it is an identifier for a note in the graph. It is
essentially a string where the values of the previous fields except the
octave are
concatenated. For instance a semi-quaver triplet F note with a bending
to G would have the resulting node key: ``05:0.042:b2''. There may be a
case in which the bending should not be considered, in this case the bending distance offset 
will be added to the base note resulting in: ``07:0.042''.
\end{itemize}

In the next section it will be shown how the data model, introduced here, is used in the
network creation.
\section{Networks Generator}
\label{sec:networks}
%TODO add citations
The network is modeled as a directed multigraph\cite{graph-theory} and it is generated
by scanning the input music sheet file. Let $G$ be a directed
multigraph defined as an ordered 4-tuple $(N,E,s,t)$ with
\begin{itemize}
	\item $N$ a set of nodes,
	\item $E$ a set of edges,
	\item $s: E \rightarrow  N$, assigning to each edge its source node,
	\item $t: E \rightarrow N$, assigning to each edge its target node.
\end{itemize}
Moreover a node $n \in N$ contains a musical note identified by its Node Key.
An edge $e \in E$ from a node $s$ to a node $t$ exists if the musical note
contained in the node $t$ was a successor, in the input music sheet, of the musical note 
contained in the node $s$. Also, there may exist more than one edge from
a node $s$ and $t$,
insofar as a same musical note can be reached by another note many times with
different octaves jumps. Thus an edge from $s$ to $t$ indicates with which
octave jump the note contained in $s$ can reach $t$ in a graph walk step.
An example of a multigraph with its corresponding sheet music is shown
in figure \ref{fig:multigraph}.
\begin{figure}
\centering
\includegraphics[scale=0.4]{multigraph.png}
\caption{A multigraph example. The corresponding input data used for the multigraph
creation is represented with its sheet. The id of each edge
indicates with which octave jump its source node can reach the target
node.}
\label{fig:multigraph}
\end{figure}
The octave informations are not included in the graph nodes because 
similar licks can be played at different octaves and otherwise it would be
harder to underline the relations between them.

It has been decided to add a weight to each edge that corresponds to how many
times an edge is passed through. This means that many walks
from a node $a$ to a node $b$ with the same distance $d$ but different
total weight exist in the graph, and the walk more likely
to correspond to a significant lick for the graph is the one with the
highest total weight. A lick is intended to be significant if it is
similar to the licks that are already used in the graph creation. The
process that determines if a lick is significant for a certain graph will
be covered in the next section.\\
A simplification of the model has been introduced in the development
process: the directed multigraph is represented as a directed graph,
where an edge is an object that contains an array which each $i_{th}$ element
indicates the pass-through frequency number for the octave corresponding
to $i$. More specifically if $f(x)$ is the frequency number related to
an octave jump $x$, then the $i_{th}$ element of the array would be:
\[ v(i) = f(i - N_o), \]
where $N_o$ is the maximum octave increment or decrement possible.
There are 5 octaves on a standard tuned guitar, thus with 6 octaves even
the 7-strings guitars and most of the non-standard tuned guitars can be
covered. Therefore the size of the array is 12 because a node $s$ can reach
in a single step a node $t$ with an increment or decrement of at most 6
octaves. To clarify, if the node $s$ has reached with a
single step the node
$t$ 5 times without any octave jump, 3 times with an octave jump of +1, 
1 time with an octave jump of +2
and 2 times with an octave jump of -1, the resulting octave jumps array
for the edge $e$ that connects $s$ and $t$ would be:
\[ v = (0,0,0,0,0,2,5,3,1,0,0,0,0) \]
 A directed edge $e$ from $s$ to $t$ exists if at least one
element of the array is non-zero.
The algorithm \ref{algo:gcreation} shows how the graph creation process
works.
\begin{algorithm}
\caption{Graph creation algorithm}
\label{algo:gcreation}
\begin{algorithmic}[1]
\Function{createGraph}{$sheets$}
	\State $G \gets $ emptyGraph()
	\For {$s$ in $sheets$}
		\For {$n$ in $s.notes$}
			\If {$pn$} 
				\State $S_k \gets $ genNodeKey($pn$)
				\State $T_k \gets $ genNodeKey($n$)
				\If {$S_k \notin G$.nodes }
					\State createNode($G$, $S_k$)
				\EndIf
				\If {$T_k \notin G$.nodes }
					\State createNode($G$, $T_k$)
				\EndIf
				\State $E_k\gets A_k + B_k$ 
					\Comment{\scriptsize{Strings concatenation}}
				\small
				\If {$E_k \notin G$.edges }
					\State createEdge($G$, $E_k$)
				\EndIf
				\State $e \gets$ getEdge($G$, $E_k$)
				\State $ojump \gets n$.octave - $pn$.octave 
				\State $id \gets ojump$ + N\_OCTAVES
				\Comment{\scriptsize{N\_OCTAVES is 6 for a guitar}}
				\small
				\State $e$.ojumps[$id$] += 1
			\EndIf
			\State $pn \gets n$
		\EndFor
	\EndFor
\EndFunction
\end{algorithmic}
\end{algorithm}

\subsection{Networks Analysis}
\begin{figure}[H]
\centering
\includegraphics[scale=0.43]{hendrix_graph.png}
\caption{A graph example created with 6 Jimi Hendrix
solos. Edges of self linked nodes are not shown.}
\label{fig:hendrixgraph}
\end{figure}
The figure \ref{fig:hendrixgraph} shows a graph created applying the
algorithm \ref{algo:gcreation} using 6 Jimi Hendrix solo files as input.
The resulting graph counts 124 nodes and 484 edges and the bending
effects have been considered. Interestingly, the same graph created
without taking the bending effects into account counts 101 nodes. This seems to
also happen with other graphs and the difference is greater for those
guitarists that frequently resort to bendings. This suggests that 
considering more effects such as vibratos or virtual harmonics increases
the vertex cardinality.

Many research articles about real network graphs, like biological, 
social and technological graphs, showed that they share a common
feature: the so-called small-world (SW) property \cite{sw-watts}.
In SW networks most nodes can be reached from every other by a small
numbers of steps and they are characterized by a high
clustering coefficient. In order to verify if guitar
licks networks respects the SW properties, the average clustering coefficient $C$ and the
average shortest path length $L$ must be determined\cite{sw-ubiquity}. In this analysis
the network is converted to a non-directed graph as a simplification. Let us start by defining 
the clustering coefficient: given $e_{ij}$ an edge from the vertex $v_i$
to the vertex $v_j$, the number of cycles of length 3 (closed triplets) from/to a node $i$ is:
\[ p_{ii}^{(3)} = \sum_{jk}{e_{ij}e_{jk}e_{ki}}\;, \]
thus the local clustering coefficient of a node $i$ is 
\[ C_i = \frac{p_{ii}^{(3)}}{k_i(k_i-1)/2} \; ,\]
and the average clustering coefficient of a graph is
\[ C = \frac{1}{N} \sum_{i}{C_i} \;  \]
where $N$ is the number of nodes in the graph.
Moreover the shortest path length average $L$, is defined as:
\[ L = \frac{1}{N(N - 1)} \sum_{i,j,i \neq j}{d_{ij}} \]
where $d_{ij}$ is the shortest path length from a node $i$ to a node
$j$.
Then the table \ref{tab:sw} shows the $C$ and $L$ values of some graphs created with the algorithm
\ref{algo:gcreation} and converted to non-directed graphs. In addition
also the $C_{rand}$
and $L_{rand}$ values are represented. $C_{rand}$ and $L_{rand}$ values
indicate the clustering coefficient and average path length of the equivalent
derived random networks. The last column in the table refers to the small-coefficient $\sigma$ defined as:
\[ \sigma = \frac{C / C_{rand}}{L / L_{rand}} \]

What is remarkable is that the $\sigma$ values are $> 1$, also $C \gg
C_{rand}$ and $L \approx L_{rand}$, satisfying the SW property
\cite{sw-ubiquity}.
\setlength{\tabcolsep}{8pt}
\begin{table}
\begin{center}
  \begin{tabular}{ l c c c c c c  }
    \hline
    graph  & $N$ & $C$ & $C_{rand}$ & $L$ & $L_{rand}$ & $\sigma$ \\ \hline
    Dave Murray & 102 & 0.329 & 0.074 & 2.864 & 2.602 & 4.039 \\
	Jimi Hendrix & 124 & 0.429 & 0.064 & 2.675 & 2.649 & 6.63 \\
	Angus Young & 122 & 0.266 & 0.068 & 3.092 & 2.888 & 3.65 \\ 
	Whole Graph & 213 & 0.531 & 0.049 & 2.734 & 2.627 & 10.4 \\
    \hline
  \end{tabular}
\end{center}
  \caption{Small World comparison results. Each graph is created with
the tablature files of a particular guitarist except for the Whole Graph
that is created with all the tablature files used for the other graphs
generation.}
  \label{tab:sw}
\end{table}
 
\section{Licks Classifier and Generator}
As introduced before, a possible application for the graphs created with
the method described in the previous section may be a lick classifier.
Let us introduce the scenario in which a graph is created with the solos
of different guitarists. It may be tempting to identify to which one of them an unknown
guitar lick belongs to. This is possible because the links in the
graphs are weighted according to the number of times they are walked
through, meaning that if a sequence of musical notes has a higher 
total weight in a walk on a graph $a$ than on a graph $b$, the probability that the sequence 
occurs more often on the graph $a$, and in the
corresponding guitarist solos, than on the graph $b$ is also higher.
The lick classifier works like this: from a set of graphs and one
lick it simply performs a walk of the lick on every graph collecting the
corresponding total weights. The total weights are then dived by the
number of notes used in the corresponding graphs creation.

Another proposed application is the Licks Generator: through the
\emph{LickNet} user
interface, it is possible to select a graph from the graphs list and
generate a set of licks which is sorted in a descending order by
the matching lick score. Its algorithm is very simple: first it
performs a very big number of random walks on the selected graph. Each
of the walk has a resulting total weight with which the generator is
able to sort them. Only the first $N$, $N$ being a small number, 
licks are then considered to be significant for the selected graph.

It will be shown in the next section that even if the two applications
described here are very simple, they can be effective.

\section{Tests and Results}
Table \ref{tab:classifier} shows the results of 4 tests
made with the lick classifier. For convenience let us call
\emph{library} the set of graphs used in the experiments. Every graph of
the library has been generated from a set of solos of a particular
guitarist, namely Jimi Hendrix, Dave Murray and Angus Young.\\
For each experiment a guitar solo (licks set that was not used in
the creation of any of the graphs) has been classified. 
Each row indicates a different
experiment, the first column indicates the guitarists of the unknown
licks set used
for each experiment and the remaining columns indicate the scores gained
for each graph by the corresponding unknown licks set. \setlength{\tabcolsep}{8pt}
\begin{table}
\begin{center}
  \begin{tabular}{ l c c c }
    \hline
    Lick guitarist  & Dave Murray & Jimi Hendrix & Angus Young  \\ \hline
	Jimi Hendrix & 0.285 & \underline{\textbf{0.356}} & 0.196 \\
    Dave Murray & \underline{\textbf{0.268}} & 0.052 & 0.040 \\
	B.B. King & 0.011 & \underline{\textbf{0.057}} & \textbf{0.045} \\
	Kirk Hammett & \textbf{0.724} & \underline{\textbf{0.785}} & 0.403 \\
  \end{tabular}
\end{center}
\caption{Licks classifier test results.}
\label{tab:classifier}
\end{table}

Moreover the scores are relative to each test and should not be
interpreted globally, as they are dependent on the length of a particular lick's notes sequence.
It is interesting how well the classifier recognizes and classifies an unknown guitarist's 
licks set. What is also remarkable is that the matching expectation of the experiments are 
satisfied: without any surprise, the licks set of B.B. King 
is a closer match to the graph of Jimi Hendrix than the others. Also it is not
very far from Angus Young while it is from the score of Dave Murray.
This may be caused by the fact that Jimi Hendrix and Angus Young have been
more influenced by the blues playing style. Furthermore it may seem
weird that the licks set of Kirk Hammett, the lead guitarist of Metallica, 
is a closer match to Jimi Hendrix than the others. But it is also close to 
Dave Murray while it is farther away from Angus Young. This may be caused by the fact that Hammett was influenced by
Jimi Hendrix in his early years \cite{hammet-hendrix} and that the genre played by Metallica
is closer to the genre played by Iron Maiden than the genre played by
AC/DC\footnote{In \cite{artists-network}, Metallica and Iron Maiden have
the same grey color(metal), while AC/DC is red colored(rock).}. In the end this is all speculation, a better interpretation would be desirable, also taking into consideration 
Kirk Hammett's graph, in order to determine the gap between those
scores and the score achieved by Kirk Hammett himself. For example by the
first two experiments one may interpret that Jimi Hendrix has a closer
relation to Dave Murray than Angus Young, but with such limited amount
of data this cannot be easily verified. A correlation metric among guitarists
using my lick classifier may be modeled in future works.


The Licks Generator has been tested with the same library. For every
guitarist a set of licks has been generated and tested again with the lick 
classifier in order to check whether the generated licks
can be traced back to the respective guitarists. As expected, each set
of generated licks gains the best score in correspondence to the right
guitarist (Table \ref{tab:generator}).\\
In addition, in the LickNet source code repository\footnote{The
repository is currently located at
\url{https://github.com/matteomartelli/licknet}. Everything is free and open source.
The instructions for contributing are provided in the README file.}
 an example data folder has been provided, containing tablature files used
to generate the library, tablature files of the licks used in the
classifier tests and tablature files of the licks created with the
lick generator.
\begin{table}
\begin{center}
  \begin{tabular}{ l c c c }
    \hline
    Lick guitarist  & Dave Murray & Jimi Hendrix & Angus Young  \\ \hline
	Jimi Hendrix & 0.220 & \underline{\textbf{0.323}} & 0.069 \\
    Dave Murray & \underline{\textbf{0.293}} & 0.208 & 0.052 \\
	Angus Young & 0.117 & 0.131 & \underline{\textbf{0.449}} \\
  \end{tabular}
\end{center}
\caption{Results of the generated licks tested with the Licks Classifier.}
\label{tab:generator}
\end{table}

\section{Future Developments}
The first one is to find a method to automatically retrieve and
organise the tablature files. The main problem in this case is that these files are
often badly written and most of them lack the key signature indication,
therefore such a method can be hard to develop. 
Another solution
may be to provide a public database of tablature files which have already
been vetted, corrected and reorganised by their human users for usage by LickNet.
As already introduced before, a second enhancement may be the
implementation of a tool for
correlating an artist with his licks and then comparing his style with
other artists in the library simply evaluating the score gained for each
corresponding graph.

Last but not least this software may be attached to other projects: for example it may be used 
in an improvisation software to generate notes for a 
soloist musician, or it may be attached to some tablature editor to easily
generate licks as starting points for compositions.

\section{Conclusions}
Concerning complex network analysis, the networks of licks seem to satisfy the small
world property. Furthermore, these networks seem to have a
structure which is highly dependant from input data and from their corresponding guitarists. 
Therefore, functions that operate on the guitarists signature playing style, such as the
licks classifier and the licks generator, stand to reason.

The initial expectations are well satisfied and LickNet appears to be a
good start point for developing different kinds of software related to
music classification and composition.

\begin{thebibliography}{50}
	\bibitem{complex-networks} 
		Maarteen van Steen, 
		\textsl{Graph Theory and Complex Networks: An Introduction}, 
		2010.
	\bibitem{music-genres}
		R. Lambiotte, M. Ausloos,
		Uncovering collective listening habits and music genres in bipartite networks,
		\textsl{Phys. Rev. E 72, 066107},
		2005
	\bibitem{artists-network}
		Dr Tamás Nepusz,
		Reconstructing the structure of the world-wide music scene with Last.fm,
		http://sixdegrees.hu/last.fm
	\bibitem{metal-network}
		Benjamin Lind,
		Patterns in the Ivy: The Small World of Metal,
		http://badhessian.org/2013/09/patterns-in-the-ivy-the-small-world-of-metal/,
		2013
	\bibitem{sw-words}
		Ramon Ferrer i Cancho, Ricard V. Solè,
		The small world of human language,
		\textsl{Proc. Royal Soc. London B, DOI: 10.1098/rspb.2001.1800}
		2001
	\bibitem{tuxguitar}
		Daniel Mantilla,
		TuxGuitar: Editorial review,
		\textsl{Software Informer},
		2014.
	\bibitem{graph-theory}
		Balakrishnan, V. K.,
		\textsl{Graph Theory},
		1997.
	\bibitem{sw-watts}
		Watts, D. J. and Strogatz, S. H.,
		Collective dynamics of `small-world’ networks,
		\textsl{Nature 393, 440-442},
		1998
	\bibitem{sw-ubiquity}
		Qawi K. Telesford, Karen E. Joyce, Satoru Hayasaka, Jonathan H. Burdette, Paul J. Laurienti
		The Ubiquity of Small-World Networks,
		\textsl{Brain Connect.; 1(5): 367–375},
		2011
	\bibitem{hammet-hendrix}
		Brian Ives,
		Hendrix At 70: “My Favorite Guitar Player Of All Time” – Kirk Hammett Of Metallica,
		\textsl{CBS Local},
		http://wcbsfm.cbslocal.com/2012/11/25/hendrix-at-70-my-favorite-guitar-player-of-all-time-kirk-hammett-of-metallica/,
		2012
	\end{thebibliography}
\end{document}
