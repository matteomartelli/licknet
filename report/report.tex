%\documentclass[a4paper,10pt]{paper}
\documentclass{llncs}
\usepackage[utf8]{inputenc}
\usepackage{hyperref}
\usepackage{graphicx}
\usepackage{wrapfig}
\usepackage{listings}
\usepackage{float}
\usepackage{bytefield}
%\usepackage{fullpage}
\usepackage{algorithm}
\usepackage{algpseudocode}
\usepackage{amsmath}
%\usepackage{todonotes}

%opening
\title{LickNet: Software Tools Collection for Guitar Licks Networks}
\author{Matteo Martelli}
\institute{
	University of Bologna\\ 
	\email{matteo.martelli9@studio.unibo.it}
}
\begin{document}

\maketitle

\begin{abstract}
\end{abstract}
\section{Introduction}
Musicians are often intrigued on how other musicians compose or
improvise their music and if there are some relations with their
previous composed music or with the compositions of other musicians.
Even non expert music enthusiasts often wonder what makes a 
musician unique, well recognisable between the others, or what makes a
musician so similar to another. For example
this frequently happens with guitarists, when people recognize 
a "Santana" guitar solo or a "Hendrix" guitar solo just listening 
 few notes and having never heard that particular solo, or when people
says that Steve Vai sounds similar to Joe Satriani.\footnote{Last.fm, a
collaborative website, assigns a "super" level of similarity between Steve Vai
and Joe Satriani. This may be related to the fact that Steve Vai was a
his student.}.
One may think that this behaviour is caused by the fact that there are
some notes patterns
repetition in one musician compositions history or that there are some
similar notes patterns between different musicians compositions caused
by their similar musical influence or musical tastes.\\
Thus with the aim to analyze
the relations between notes patterns in guitar
solos, often called guitar licks, the developing process of the software 
described in this
document, \emph{LickNet}, has been started. The choose of guitar solos has been taken 
because of the large amount of data available online but the subject may 
be extended to
different, but similar, kind of data. For example drum patterns
relations in jam sessions or other instruments executions, compositions
and improvisations may be considered. It has been chosen to represent
the data elements and their relations within a network, as to the
means offered by the graph theory useful for a mathematical analysis.\\
The area of scientific research
that usually is interested in the empirical examination of real-world network
 is the study of \emph{Complex Networks}. Some of the study approaches
and mathematical methods that usually interests the Complex Networks
area, such as graph theory or scale-free network
comparison\cite{complex-networks}, 
have been considered and will be discussed in the next sections.\\
In addition to the network analysis tools, LickNet provides some other
features focused more on the non-technical use of the application, like
the \emph{Lick Classifier} and the \emph{Lick Generator}. These
functionalities are still in an early stage but can be a starting point
to realize some practical application focused to end users who may want to
understand and replicate some other guitarist playing style, or to
realize some artificial agents that may "play" some music while interacting
with human players.
\section{Complex Networks}
\subsection{Related Works}

\section{LickNet}
As introduced before, LickNet offers the possibility to study the
relations between notes patterns in guitar solos through the creation
and analysis of guitar licks networks. Moreover it collects
some software tools with which is possible to do some particular
operations with guitar licks through the use of the created networks.
These operations are currently two, a classifier and a generator of
guitar licks.\\
Before getting into the software components, the next section is going
to focus on how the interested data are collected, structured and processed. 

\section{Data}
As many guitarists may already know, there are plenty of websites that
store guitar sheet music files. One of them is \url{www.ultimate-guitar.com}
that counts more than 800000 sheet music files, which are mostly written
with the guitar tablature notation. Guitar players usually choose this
notation for its ease of use \footnote{Guitar tablature removes the
requirement for the player to remember the associations between the
notes and the corresponding fretboard positions, as the latter are directly
represented in the tablature notation.}
but it is instrument-specific. Also guitar tablature is not standardized and different 
sheet-music publishers adopt different conventions. This means that a
guitar tablature can be understood only by guitar players and its
conversion to the standard notation or formal interpretation may result
wearing. For example the semantic of a guitar tablature changes even if
a guitar it is not standard tuned.\\
Fortunately there are various computer programs available for writing
tablature. One of most frequently used is \emph{Guitar Pro} and the
\emph{Ultimate
Guitar} servers store a large amount of tablature files encoded with its format.
Another interesting software is \emph{TuxGuitar}, a free and open source
tablature editor that also supports the ability to import and export
Guitar Pro files\cite{tuxguitar}. Moreover its source code can be freely
re-used and adapted to any other application as it has been done with
LickNet with the purpose of importing the tablature files retrieved
online.\\
After that the tablature can be interpreted by software, a guitar solo
is represented as a sequence of notes, where each of them is composed
 of the following fields:
\begin{itemize}
\item string ..
\item value ..
\end{itemize}

\section{Networks Generator}
%TODO: multigraph labeled -> graph labeled
\subsection{Complex Networks Analysis}

\section{Lick Classifier}
\subsection{Tests and Results}
%TODO show that with more guitar techniques influences the matching is better (with dave muray and bending for example)

\section{Licks Generator}
\subsection{Tests and Results}

\section{Development Complications}
%TODO tabs are shitty written!!!!!!

\section{Future Development}
%TODO artists co-relations with licks: get a lick of an artist and check
%the weights on all the others graph


\section{Conclusions}

\begin{thebibliography}{50}
	\bibitem{complex-networks} 
		Maarteen van Steen, 
		\textsl{Graph Theory and Complex Networks: An Introduction}, 
		2010.
	\bibitem{tuxguitar}
		Daniel Mantilla,
		TuxGuitar: Editorial review,
		\textsl{Software Informer},
		2014.
\end{thebibliography}
\end{document}
